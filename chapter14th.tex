%!TEX program = xelatex
\documentclass[a4paper]{article}
\usepackage{fontspec,xunicode,xltxtra}
\usepackage{geometry}

\setmainfont{Hiragino Sans GB}
\title{维纳过程和伊藤引理}
\author{}

\begin{document}
\maketitle{}
\begin{center}
如果一个变量的值以某种不确定的形式随时间变化,我们称这个变量服从某种随机过程(stochastic process)。随机过程可以分为离散时间(discrete time)和连续时间(continuous time)两类:一个离散时间随机过程是指变量值只能在某些确定的时间点上变化,
而一个连续时间随机过程是指变量值可以在任何时刻上变化,从而随机过程也可以被分为连续变量(continous varibale)和离散变量(discrete variable)两类。
\end{center}
\section{马尔科夫过程}
马尔科夫过程是一个特殊类型的随机过程,其中只有标的的变量的当前值与未来的预测有关,\\
而变量的历史值以及变量从过去到现在的演变方式与未来的预测无关。\\
由此我们可以得出马尔科夫过程实际上不依赖于历史的假设,只是根据当前观察到的现象给出一个预测的概率。\\



\end{document}

